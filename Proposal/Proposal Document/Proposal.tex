%%%%%%%%%%%%%%%%%%%%%%%%%%%%%%%%%%%%%%%%%
% Sullivan Business Report
% LaTeX Template
% Version 1.0 (May 5, 2022)
%
% This template originates from:
% https://www.LaTeXTemplates.com
%
% Author:
% Vel (vel@latextemplates.com)
%
% License:
% CC BY-NC-SA 4.0 (https://creativecommons.org/licenses/by-nc-sa/4.0/)
%
%%%%%%%%%%%%%%%%%%%%%%%%%%%%%%%%%%%%%%%%%


%----------------------------------------------------------------------------------------
%	CLASS, PACKAGES AND OTHER DOCUMENT CONFIGURATIONS
%----------------------------------------------------------------------------------------

\documentclass[
    a4paper, % Paper size, use either a4paper or letterpaper
	12pt, % Default font size, the template is designed to look good at 12pt so it's best not to change this
	%unnumberedsections, % Uncomment for no section numbering
    ]{CSSullivanBusinessReport}
    
    \addbibresource{sample.bib} % BibLaTeX bibliography file

%----------------------------------------------------------------------------------------
%	REPORT INFORMATION
%----------------------------------------------------------------------------------------

\reporttitle{CPE 133 Final Lab Proposal} % The report title is to appear on the title page and page headers, do not create manual new lines here as this will carry over to page headers

\reportsubtitle{Robotic Arm\\ with IR input and Servo Output V2} % Report subtitle, include new lines if needed

\reportauthors{Proposed by:\\\smallskip Ethan Vosburg (evosburg@calpoly.edu) \\Wyatt Tack (wtack@calpoly.edu)} % Report authors/group/department, include new lines if needed

\reportdate{\today} % Report date, include new lines for additional information if needed

\rightheadercontent{\includegraphics[width=3cm]{creodocs_logo.pdf}} % The content in the right header, you may want to add your own company logo or use your company/department name or leave this command empty for no right header content

%----------------------------------------------------------------------------------------

\begin{document}

%----------------------------------------------------------------------------------------
%	TITLE PAGE
%----------------------------------------------------------------------------------------

\thispagestyle{empty} % Suppress headers and footers on this page

\begin{fullwidth} % Use the whole page width
	\vspace*{-0.075\textheight} % Pull logo into the top margin
	
	\hfill\includegraphics[width=5cm]{creodocs_logo.pdf} % Company logo

	\vspace{0.15\textheight} % Vertical whitespace

	\parbox{0.9\fulltextwidth}{\fontsize{50pt}{52pt}\selectfont\raggedright\textbf{\reporttitle}\par} % Report title, intentionally at less than full width for nice wrapping. Adjust the width of the \parbox and the font size as needed for your title to look good.
	
	\vspace{0.03\textheight} % Vertical whitespace
	
	{\LARGE\textit{\textbf{\reportsubtitle}}\par} % Subtitle
	
	\vfill % Vertical whitespace
	
	{\Large\reportauthors\par} % Report authors, group or department
	
	\vfill\vfill\vfill % Vertical whitespace
	
	{\large\reportdate\par} % Report date
\end{fullwidth}

\newpage

%----------------------------------------------------------------------------------------
%	DISCLAIMER/COPYRIGHT PAGE
%----------------------------------------------------------------------------------------

% \thispagestyle{empty} % Suppress headers and footers on this page

% \begin{twothirdswidth} % Content in this environment to be at two-thirds of the whole page width
% 	\footnotesize % Reduce font size
	
% 	\subsection*{Disclaimer}

% 	Lorem ipsum dolor sit amet, consectetur adipiscing elit. Praesent porttitor arcu luctus, imperdiet urna iaculis, mattis eros. Pellentesque iaculis odio vel nisl ullamcorper, nec faucibus ipsum molestie. Sed dictum nisl non aliquet porttitor. Etiam vulputate arcu dignissim, finibus sem et, viverra nisl. Aenean luctus congue massa, ut laoreet metus ornare in. Nunc fermentum nisi imperdiet lectus tincidunt vestibulum at ac elit.
	
% 	\subsection*{Copyright}
	
% 	\textcopyright~[Year] [Company] 
	
% 	Copyright notice text\ldots In hac habitasse platea dictumst. Curabitur mattis elit sit amet justo luctus vestibulum. In hac habitasse platea dictumst. Pellentesque lobortis justo enim, a condimentum massa tempor eu. Ut quis nulla a quam pretium eleifend nec eu nisl. Nam cursus porttitor eros, sed luctus ligula convallis quis.
	
% 	\subsection*{Contact}
	
% 	Address Line 1\\
% 	Address Line 2\\
% 	Address Line 3
	
% 	Business Number 123456
	
% 	Contact: name@company.com
	
% 	\vfill % Push the following down to the bottom of the page
	
% 	\subsubsection*{Changelog}
	
% 	\scriptsize % Reduce font size further
	
% 	\begin{tabular}{@{} L{0.05\linewidth} L{0.15\linewidth} L{0.6\linewidth} @{}} % Column widths specified here, change as needed for your content
% 		\toprule
% 		v1.0 & 20XX-02-05 & Lorem ipsum dolor sit amet, consectetur adipiscing elit. Praesent porttitor arcu luctus, imperdiet urna iaculis, mattis eros.\\
% 		v1.1 & 20XX-02-27 & Pellentesque iaculis odio vel nisl ullamcorper, nec faucibus ipsum molestie.\\
% 		v1.2 & 20XX-03-15 & Sed dictum nisl non aliquet porttitor.\\
% 		\bottomrule
% 	\end{tabular}
% \end{twothirdswidth}

% \newpage

%----------------------------------------------------------------------------------------
%	TABLE OF CONTENTS
%----------------------------------------------------------------------------------------
\bigskip
\begin{twothirdswidth} % Content in this environment to be at two-thirds of the whole page width
	\tableofcontents % Output the table of contents, automatically generated from the section commands used in the document
\end{twothirdswidth}

\newpage

%----------------------------------------------------------------------------------------
%	SECTIONS
%----------------------------------------------------------------------------------------

\section{Project Description} % Top level section

This project proposes the development of a robotic arm with a three-degrees-of-freedom (3DOF)\sidenote{This refers to the number of rotational axes the robot will have, in this case, 3 axes} movement capability, enabling it to maneuver in three-dimensional space. The core of this system is its ability to receive inputs through an infrared (IR) transmitter, making it responsive to remote commands. The IR signals will be routed into an FSM capable of sequence detection to discern and interpret the various signals it receives. This ensures that each command is distinctly recognized, allowing for a seamless flow from remote input to mechanical action.

Upon detecting an input, the IR receiver\sidenote{Commands will be sent to the receiver via an IR remote control} decodes the signal to determine the intended operation. The decoded instructions are then used to manipulate the robotic arm, with servos that actuate movement. These servos will be controlled via PWM signals generated by a decoder/pulse-width modulation (PWM) module, which adjusts the clock timings to control the servos with high precision. The integration of these items will be completely done in the FPGA on the board. 


\section{High Level Black Box Diagram} % Second level section

\begin{figure}[h]
    \centering
    \includegraphics[width=.8\textwidth]{Pictures/HLBBD.png}
    \caption[center]{High-Level Block Diagram}
    \label{fig:highlevelblockdiagram}
\end{figure}
\newpage
\section{Low Level Diagram} % Third level section


\begin{fullwidth} % Use the whole page width
    \begin{figure}
        \centering
        \includegraphics[width=.8\pdfpagewidth]{Pictures/High Level Diagram V2.png}
        \caption{Low-Level Block Diagram}
        \label{fig:lowlevelblockdiagram}
    \end{figure}
    % Indicate in your low-level diagram how the FSM, accumulator and new module criteria are met
	As noted in the diagram the FSM is implemented in the IR Decoder module. Multiple accumulators are used to slow down the command rate as well as store the current location of the robotic arm. The new module criteria are met by the decoder module that sends signals to the accumulators as well as the decoders that send the PWM signal out. There will be multiple modules used including new ones that will have to be developed.
\end{fullwidth}

% \beg 


%----------------------------------------------------------------------------------------

\end{document}
